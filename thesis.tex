\documentclass[a4paper,10pt]{article}
\usepackage[utf8]{inputenc}
\usepackage[T1]{fontenc}

% This package provides a flexible and easy interface to page dimensions.
% You can change the page layout with intuitive parameters.
% For instance, if you want to set a margin to 2cm from each edge of the paper,
% you can type just \usepackage[margin=2cm]{geometry}.
% The page layout can be changed
% in the middle of the document with \newgeometry command.
% See geometry  documentation file geometry.pdf,
% v5.8, 2018/04/16, Abstract, page 1.
%
% The default margins depend on the current paper size
% and are shown in the log file.
% See:
% https://tex.stackexchange.com/a/44850
% or
% https://tex.stackexchange.com/questions/44846/
% default-margins-for-geometry-package/44850#44850
\usepackage{geometry}

\usepackage{url}

% Use french typography rules.
\usepackage[
  % In multilingual documents, just use several options.
  % The last one is considered the main language, activated by default.
  % Sometimes, the main language changes the document layout
  % (eg, spanish and french).
  % You can also set the main language explicitly:
  % \documentclass{article}
  % \usepackage[main=english,dutch]{babel}
  % See babel documentation file babel.pdf,
  % Version 3.31, 2019/05/04, section 1.2, page 5.
  english,
  main=french,
]{babel}

% \babeltags{<tag1> = <language1>, <tag2> = <language2>, ...}
% New 3.9i In multilingual documents with many language switches
% the commands above can be cumbersome.
% With this tool shorter names can be defined.
% It adds nothing really new – it is just syntactical sugar.
% It defines \text<tag1>{<text>} to be \foreignlanguage{<language1>}{<text>},
% and \begin{<tag1>} to be \begin{otherlanguage*}{<language1>}, and so on.
% Note \<tag1> is also allowed, but remember to set it locally inside a group.
% See babel documentation file babel.pdf,
% Version 3.31, 2019/05/04, section 1.9, page 10.
\babeltags{english=english, french=french}

% Add nice quotation macros.
\usepackage[
  % This option controls multilingual support.
  % It requires either the babel package or the polyglossia package.
  % autostyle=true continuously adapts the quote style
  % to the current document language;
  % once will only adapt the style once so that it matches
  % the main language of the document.
  % autostyle=try and tryonce are similar
  % to true and once if multilingual support is available
  % but will not issue any warnings
  % if not (i.e., if neither babel nor polyglossia have been loaded).
  % The short form autostyle is equivalent to autostyle=true.
  % See csquotes documentation file csquotes.pdf,
  % v5.2e, 2019/05/10, section 2.1, page 2-3.
  autostyle=true,
]{csquotes}

% \sloppy

\begin{document}

\title{\textbf{Proposition de Thèse Paris-Saclay/LRI}}
\author{Idir Ait Sadoune et Burkhart Wolff}
\date{}

\maketitle

\begin{english}

\subsection*{Title}

\begin{center}
  \large An Ontology Framework for Formal Libraries
\end{center}

\end{english}

\subsection*{Titre}

\begin{center}
  \large Conception et Implémentation d'un Environment d'Ontologie
    pour des Bibliothèques Formelles
\end{center}

\begin{english}

\subsection*{Keywords}

Ontologies \textemdash{}
Semantic Web \textemdash{}
Formal Theories \textemdash{}
Interactive Proof Development \textemdash{}
Formal Program Development \textemdash{}
Access to Formal Knowledge

\subsection*{Detailed Thesis Overview}

With the maturation and growing power of interactive proof systems,
the body of formalized mathematics and engineering is dramatically increasing.
Documents containing mixed forms of formal and informal content 
represent a particular challenge for linking these forms of information:
texts may contain formulas (which should be at least type-correct and 
if possible semantically consistent with the formal definitions) 
and formal definitions based on naming conventions should reflect
informal explanations. Keeping this information in sync and
using it for interaction and search represents a particular problem.
This challenge incites numerous research efforts
summarized under the labels \enquote{semantic web}, \enquote{data mining},
or any form of advanced \enquote{semantic} text processing.
A key role in structuring this linking is played by document ontologies,
i.e., a machine-readable form of the structure of documents
as well as the document discourse.
In many scientific disciplines such as medicine or biology,
ontologies play a crucial role in the organization of research papers
and their automated access.

While it is desirable to have analogous techniques
in the field of mathematics and engineering,
there are particular challenges in this the field:
this type of documents contains both formal and informal text-elements
with a complex structure of mutual links and dependencies of various types.
A deeper access into the formal parts of this type of documents involves
a framework that can cope with typed, logical languages,
which requires going substantially further
than existing ontological languages
like, for example, OWL or R~\cite{owl2012,protege,owlgred,rontorium}.

A first approach for the class of ontological languages
addressing this challenge
both for mathematical and formal engineering texts is
Isabelle\_DOF
\cite{
  Brucker-ea.Using-CICM18,
  BruckerWolff.Design-SEFM19,
  BruckerWolff.Certif-IFM19%
}.
Deeply integrated into the interactive theorem proving system Isabelle/HOL
and its front-end \emph{PIDE},
it allows both the development of simply-typed ontologies
and of documents containing definitions, documentation,
and formal proofs.
The ontologies provide structured meta-information and structural constraints
which were enforced during document editing.
However, the current implementation is document-centric,
i.e., it supports links to terms in definitions and proofs,
but not links between terms, nor the addition of structured meta-information
\emph{inside and between} formulas or proofs.
These features are nevertheless essential for a number of applications
when it comes to the exchange of (semi-)formal information
between interactive and automated provers on the one hand
and to advanced \enquote{semantic}, knowledge-oriented search techniques
in these documents on the other hand.

This thesis will overcome this limitation:
A new language will be designed and implemented
(possibly, but not necessarily based on Isabelle\_DOF),
that will describe and enforce meta-information
inside the level of terms and proof-objects.
It is therefore designed to provide a \enquote{deeper} integration
into formal mathematical library texts,
enabling both tool interaction and knowledge-oriented access
for advanced search in libraries such as the AFP, MMT, TheMizarJournal
or Dedukti/Logipedia~\cite{afp,mmt,TheMizarJournal,dedukti}.
In particular, it is intended to coordinate this
with the Logipedia Initiative,
which currently submitted an EU Research Funding Proposal
(project coordinators: Gilles Dowek, Frederic Blanqui).

\end{english}

\subsection*{Présentation détaillée du projet doctoral}

Avec la maturation et la puissance croissante
des systèmes de preuve interactifs,
le corps des mathématiques et de l'ingénierie formelles augmente considérablement.
Ce défi incite à de nombreux efforts de recherche
rassemblés sous les bannières du \enquote{Web sémantique},
de l'\enquote{exploration de données}
ou de toute autre forme de traitement de texte \enquote{sémantique} avancé.
Un rôle clé dans la structuration de ce lien est joué
par les ontologies de document,
c'est-à-dire une forme lisible par la machine de la structure des documents
et de la matière des documents.
Dans de nombreuses disciplines scientifiques comme la médecine ou la biologie,
les ontologies jouent un rôle crucial
dans l'organisation des articles de recherche et leur accès automatisé.

S'il est souhaitable d'avoir des techniques analogues
dans le domaine des mathématiques et de l'ingénierie,
il existe des défis particuliers dans ce domaine :
ce type de documents contient des éléments de texte formels et informels
avec une structure complexe de liens mutuels et de dépendances de divers types.
Un accès approfondi aux parties formelles de ce type de documents implique
un cadre capable de faire face à des langages logiques typés,
ce qui nécessite d'aller beaucoup plus loin
que les langages ontologiques existants,
par exemple OWL ou R \cite {owl2012,protege,owlgred,rontorium}.

Une première approche pour la classe des langages ontologiques
s'attaquant à ce défi
à la fois pour les textes d'ingénierie mathématiques et formels est
Isabelle\_DOF
\cite{
  Brucker-ea.Using-CICM18,
  BruckerWolff.Design-SEFM19,
  BruckerWolff.Certif-IFM19%
}.
Pleinement intégré
au système interactif de démonstration de théorèmes Isabelle/HOL
et à son interface \emph{PIDE},
il permet à la fois le développement d'ontologies simplement typées
et de documents contenant des définitions, de la documentation
et des preuves formelles.
Les ontologies fournissent des méta-informations structurées
et des contraintes structurelles
qui ont été appliquées lors de l'édition des documents.
Cependant, l'implémentation actuelle est centrée sur le document,
c'est-à-dire qu'elle prend en charge les liens vers les termes
dans les définitions et les preuves,
mais pas les liens entre les termes,
ni l'ajout de méta-informations structurées
\emph{à l'intérieur et entre} formules ou preuves.
Ces fonctionnalités sont néanmoins
essentielles à un certain nombre d'applications
pour l'échange d'informations (semi-)formelles
entre des prouveurs interactifs et automatisés d'une part,
et pour des techniques de recherche \enquote{sémantique} avancée,
orientées connaissances, dans ces documents d'autre part.

Cette thèse surmontera cette limitation :
un nouveau langage sera conçu et implémenté
(éventuellement, mais pas nécessairement fondé sur Isabelle\_DOF),
qui décrira et appliquera les méta-informations
à l'intérieur du niveau des termes et des objets de preuve.
Il est donc conçu pour fournir une intégration \enquote{plus profonde}
dans des textes de bibliothèques mathématiques formelles,
permettant à la fois une interaction avec les outils
et un accès axé sur les connaissances
pour une recherche avancée dans des bibliothèques
telles que AFP, MMT, TheMizarJournal ou Dedukti / Logipedia
\cite {afp,mmt,TheMizarJournal,dedukti}.
En particulier, il est prévu coordonner cela
avec l'Initiative Logipedia,
qui vient de soumettre une proposition de projet dans le cadre de H2020
(coordinateurs du projet : Gilles Dowek, Frederic Blanqui).

\subsection*{Thématique}

Le lien entre le formel et l'informel est peut-être
le défi le plus répandu dans la numérisation des connaissances
et leur propagation.
Ce défi implique de nombreux efforts de recherche
rassemblés sous les bannières du \enquote{Web sémantique},
de l'\enquote{exploration de données}
ou de toute autre forme de traitement de texte \enquote{sémantique} avancé.
Un rôle clé dans la structuration de ce lien est joué
par les ontologies de document
(également appelé vocabulaire dans la communauté du Web sémantique
\cite{owlgred}),
c'est-à-dire une forme lisible par la machine de la structure des documents
et de la matière des documents.
De telles ontologies peuvent être utilisées
par le discours scientifique dans les articles savants,
les bibliothèques mathématiques
et par le discours d'ingénierie des documents normalisés
de certification de logiciels.

Avec la maturation et la puissance croissante
des systèmes de preuve interactifs,
le corps des mathématiques et de l'ingénierie formelles a augmenté
de manière considérable.
Un exemple pour cette tendance sont
les Archives de Preuves Formelles (AFP) d'Isabelle~\cite{afp} :
créés en 2004, elles comptaient au total 215 articles en 2015,
et 413 trois ans plus tard.
Une analyse empirique approfondie montre que
la complexité et la taille ont augmenté en conséquence
\cite{DBLP:conf/mkm/BlanchetteHMN15}.
Grâce à une collaboration avec l'AFP,
le nombre de documents a augmenté avec l'ajout d'articles
sur les questions formelles d'ingénierie logicielle
telles que les définitions de langue normalisées,
les structures de données, les processeurs,
les spécifications liées à la sécurité  ou les systèmes d'exploitation.
Des systèmes comme MMT~\cite{mmt}, OpenMath~\cite{openmath}
et Dedukti~\cite{dedukti} représentent
d'autres exemples de cette tendance.

\subsection*{Domaine}

Méthodes Formelles \textemdash{}
Systèmes de Preuve \textemdash{}
Mathematical Content Management \textemdash{}
Ontologies \textemdash{}
Domain-Specific Languages (DSL)
pour l'organisation des connaissances formelles,
fouilles \enquote{sémantiques} dans des bibliothèques mathématiques.

\subsection*{Contexte}

Encadrement dans le Laboratoire de Recherche en Informatique (LRI)
dans l'équipe VALS\@.

Directeur Prof.\ B.\ Wolff (HDR), Co-encadrement : Dr.\ Idir Ait Sadoune.

Financement de Thèse : Bourse Ministérielle ou Projet Européen Logipedia.

On note que l'équipe VALS va être transférée
dans le nouveau laboratoire LMF (également Université Paris-Saclay)
pendant la durée de la thèse ;
on estime que ce transfert de l'organisation n'aura aucun impact
sur le déroulement de la thèse et les engagements contractuels.

Les encadrants Pr.\ B.\ Wolff (HDR) et Dr.\ Idir Ait Sadoune sont impliqués
dans le domaine des ontologies pour des modèles/textes formels
de multiples manières ;
B.\ Wolff est auteur du système Isabelle\_DOF,
Idir Ait Sadoune auteur de plusieurs publications dans le domaine
\cite{DBLP:conf/medi/Ait-SadouneM19}.

Le projet européen \emph{Logipedia},
à base d'un langage de preuve formelle Dedukti~\cite{dedukti}, vise
à définir une lingua franca
pour échanger des termes, preuves, théories et documentations
entre différentes bibliothèques
et à prévoir des mécanismes de fouille avancés
dans ces bibliothèques combinées.

\subsection*{Méthode}

Implémentation, développement de nouvelles ontologies
en s'appuyant sur des ontologies existantes.

Études de Cas suivant le contexte
(soit Génie Logiciel, soit Mathématique dans le contexte Logipedia).

\subsection*{Résultats attendus \textendash{}  Objectifs}

\begin{itemize}
  \item
    Conception et Implémentation d'un système comprenant
    une langage et un mécanisme de validation
    (soit par extension d'Isabelle\_DOF
    ou par le portage vers le framework Dedukti) ;
  \item
    Développement des bibliothèques d'ontologies
    pour le scénario \enquote{proof exchanges} ;
  \item
    Développement des bibliothèques d'ontologies du domaine
    pour le scénario \enquote{fouille} dans des textes
    de Mathématique and Génie Logiciel
    (par exemple une ontologie
    d'une Certification comme Critères Communs ou CENELEC) ;
  \item
    Publications dans SEFM, IFM, FM, CICM, MKM\@.
\end{itemize}

Ceci représente aussi en gros le plan de travail.

\bibliographystyle{unsrt}
\bibliography{biblio,local}

\end{document}
