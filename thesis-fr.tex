\documentclass[a4paper,10pt]{article}
\usepackage[utf8]{inputenc}
\usepackage[T1]{fontenc}
\usepackage{url,french}

\sloppy

\begin{document}

\title{\textbf{Proposition de Thèse Paris-Saclay/LRI}}
\author{Idir Ait Sadoune et Burkhart Wolff}
\date{}

\maketitle

\subsection*{Titre~:}
\begin{center}
\large  \\
  An Ontology Framework for Formal Mathematical Libraries 
\end{center}


% With the maturation and growing power of interactive proof systems, the body of formalized mathematics and engineering is dramatically increasing. The Isabelle Archive 
% of Formal Proof (AFP) [6], created in 2004, counted in 2015 a total of 215 articles, whereas the count stood at 413 only three years later. An in- depth empirical 
% analysis shows that both complexity and size increased accord- ingly [11]. Together with the AFP, there is also a growing body on articles concerned with formal 
% software engineering issues such as standardized lan- guage definitions (e. g., [15, 21]), data-structures (e. g., [14, 24]), hardware-models (e.g., [20]), 
% security-related specifications (e.g., [13,26]), or operating systems (e.g., [22,27]).
% ⃝c Springer Nature Switzerland AG 2019
% P. C. Ölveczky and G. Salaün (Eds.): SEFM 2019, LNCS 11724, pp. 275–292, 2019. https://doi.org/10.1007/978-3-030-30446-1_15
% 276 A. D. Brucker and B. Wolff
% This development raises interest in at least two ways: First, there is a sub- stantial potential of retrieve and reuse of formal developments, and second, formal 
% techniques allow a deeper checking of documents containing formal spec- ifications, proofs and tests. This paves the way for collaborative, continuously 
% machine-checked developments of certification documents involving both formal as well of informal content evolution.

\subsection*{Sujet~:}
Avec la maturation et la puissance croissante des systèmes de preuve interactifs, le corps des mathématiques et de l'ingénierie formelles s'est augmente 
de manière considérable. Un example pour cette tendence sont les 
archives d'Isabelle \cite{} de la preuve formelle (AFP) [6]: Créés en 2004, ils comptaient en 2015 215 articles au total, contre 413 seulement trois ans plus tard. 
%Une analyse empirique approfondie montre que la complexité et la taille ont augmenté en conséquence [11]. 
En collaboration avec l'AFP, il existe également un nombre croissant d'articles sur les questions formelles d'ingénierie logicielle telles que les définitions de langue normalisées (par exemple, [15, 21]), les structures de données ([14, 24]), des processeurs (par exemple, [20]), les spécifications liées à la sécurité ([13,26]) ou des systèmes d'exploitation (par exemple, [22,27]). Des systemes comme MMT\cite{XXX}, Openmath\cite{XXX} et Logipedia \cite{XXX} representent autres examples
pour cette tendance.

Le projet europeen \emph(Logipedia}\cite{XXX}, a base d'un language de preuve formelle Dedukti\cite{XXX}, vise a definir une lingua franca pour échanger des termes,
preuves et theories de ces differentes bibliotheques et de prevoir des mechanismes de fouille avancés dans ces bibliothèques combineés. 

For both --- the combination of "mathematical content" generated from different proof systems as well as advanced, domain-specific 
search technologies --- the construction of a flexible mechanism to specify and manage meta-data is fundamental importance. Since 
the precise structure of meta-data will evolve over time, a \emph{language} and a \emph{framework to enforce} the validity of meta-data 
is required. For such languages, the term  \emph{ontology} has been established in semantic web communities[3], and there are plethora 
of approaches and languages ranging from  validations of \verb*dtd*'s over XML documents to languages such as OWL \cite{}. 

However, enforcing a structure in a general document is different from stucture in mathematical libraries consisting of formal proofs.
While done by annotations with meta-information, these annotations are a machine-readable form of the structure of proofs and theories 
in terms of  an ontological discourse. Ontologies in this context must be strongly typed languages with mechanisms to reference and agglomerate  
definitions, data-structure elements, rules and rule-types (induction/co-induction), proof-methods, lemmas and proof-elements, etcpp.
With respect to search, it is moreover relevant to specify \emph{domain-specific} mathematical meta-knowledge in order to allow requests 
like "which theorems here are relevant for sampling analogous signals" or "which part of the linear algebra libraries is relevant for
specifying the amount of 'blur' in a digital image" ?

MORE TO COME: Isabelle/DOF is a first language of this type. 
... see SEFM Paper. 

\subsection*{Objectifs~:}

TODO

\begin{itemize}
\item
  Porting Isabelle/DOF to Dedukti
\item
  Developping onto libraries for proof exchanges
\item
  Developing onto libraries for domain-specific Ontologies in Math and Engineerinfg
\item
  Developping a framework of semantic links of ontologies via proofs.
\end{itemize}

\subsection*{Plan de Travail~:}
\begin{itemize}
  \item
    Modélisation des patrons \emph{open world}
  \item
    Modélisation des patrons \emph{open world}
  \item
    Développement d'un système de transformation
  \item :...
\end{itemize}

\subsection*{Cadre d'organisation~:}
Encadrement dans le laboratoire de recherche en informatique (LRI)
dans l'équipe VALS\@.
Directeur Prof.\ B. Wolff (HDR), Co-encadrement : Dr.\ Idir Ait-Sadune.
Financement de Thèse: EU Projet Logipedia.
On note que l'équipe VALS va être transférée
dans le nouveau laboratoire LMF (également Université Paris-Saclay)
pendant la durée de la thèse ;
on estime que ce transfert de l'organisation n'aura aucun impact
sur le déroulement de la thèse et les engagements contractuels.

\bibliographystyle{unsrt}
\bibliography{biblio,local}

\end{document}
