\documentclass[a4paper,10pt]{article}
\usepackage[utf8]{inputenc}
\usepackage[T1]{fontenc}
\usepackage{url,french}

\sloppy

\begin{document}

\title{\textbf{Proposition de Thèse Paris-Saclay/LRI}}
\author{Safouan Taha et Burkhart Wolff}
\date{}

\maketitle

\subsection*{Titre~:}
\begin{center}
\large Vérification et Simulation des Comportements pour la Sûreté \\
  de Fonctionnement des Systèmes Autonomes Critiques
\end{center}

\subsection*{Sujet~:}
Les systèmes autonomes ont la particularité d'être soumis,
en plus des défaillances physiques classiques,
à des erreurs relevant d'interactions avec l'environnement
ou du traitement des données
(par exemple, l'éblouissement ou une mauvaise reconnaissance de forme).

Dans de tels environnements hostiles et difficiles à modéliser,
on s'intéresse à des \emph{modèles de comportement}
dans lesquels le système est censé assurer des propriétés de sécurité
malgré la présence de données erronées ou imprécises.
La norme ISO SOTIF \cite{iso-2019-sotif} introduit
une classification de scénarios en plusieurs catégories,
\emph{known safe}, \emph{known unsafe}, \emph{unknown safe}
et \emph{unknown unsafe}, selon que le scénario est connu
lors de la conception du système ou bien découvert lors de la phase de test
et selon que le scénario ne déstabilise pas le système
ou bien provoque sa défaillance.

L'objectif de cette thèse est de trouver des modèles comportementaux
qui sont suffisamment flexibles et \emph{open world} pour prendre en compte
des scénarios \emph{known unsafe} et \emph{unknown unsafe}.

Une théorie de référence pour étudier les modèles comportementaux est
\emph{Concurrent Sequential Processes} (CSP)
qui a été introduite dans un livre en 1978
par Tony Hoare \cite{Hoare:1985:CSP:3921},
et qui a évolué de manière substantielle
entre-temps \cite{BrookesHR84,brookes-roscoe85,roscoe:csp:1998}.
CSP décrit le non-déterminisme, la communication
et la synchronisation avec un ensemble minimal d'opérateurs
et un ensemble des règles.
CSP offre à la fois un cadre d'étude théorique,
un langage de modélisation de comportement,
et un framework de vérification et simulation
via une implémentation des processus par des automates.
Les processus CSP sont ainsi décrits dans un modèle abstrait
avec une grande \emph{expressivité} et conçu pour être \emph{compositionnels}.

Isabelle/HOL-CSP \cite{HOL-CSP-AFP} est une théorie générale de CSP
implémentée dans l'environnement de modélisation et de preuve
Isabelle/HOL \cite{nipkow.ea:isabelle:2002}.
Isabelle/HOL-CSP permet entre autres d'exprimer :
%
\begin{itemize}
  \item
    des non-déterminismes non-bornés
    (ce qui correspond aux scénarios \emph{known unsafe}) ;
  \item
    une hiérarchisation des événements \emph{open-world}
    (ce qui correspond aux scénarios \emph{unknown unsafe}) ;
  \item
    des patrons de processus (ce qui correspond au \emph{known unsafe})\ldots
\end{itemize}
%
dans un cadre permettant modélisation, simulation et preuve.
En particulier, il permet de prouver de manière incrémentale des propriétés
sur des environnements avec une grande variabilité
et de vérifier la correction des processus par raffinement.

\subsection*{Objectifs~:}
\begin{itemize}
  \item
    Étude des modèles \emph{open world} d'environnements liés
    au domaine du Véhicule Autonome,
    ainsi que des patrons des systèmes de contrôle du Véhicule Autonome ;
\item
  Construction d'un environnement de simulation et de vérification
  à base de processus HOL-CSP pour le domaine du Véhicule Autonome ;
\item
  Construction d'un environnement de simulation
  à base d'abstractions correctes des comportements en HOL-CSP ;
\item
  Intégration de l'environnement résultant
  avec le système d'ingénierie AltaRica ou similaire.
\end{itemize}

\subsection*{Plan de Travail~:}
\begin{itemize}
  \item
    Modélisation des patrons \emph{open world}
    d'environnement du Véhicule Autonome ;
  \item
    Modélisation des patrons \emph{open world}
    des systèmes de contrôle du Véhicule Autonome ;
  \item
    Développement d'un système de transformation
    des processus CSP en Automates ;
  \item
    Développement d'un système de simulation des processus CSP
    et leur intégration dans un framework similaire à AltaRica ;
  \item
    Développement d'un dispositif d'abstraction des processus
    à partir des ontologies du domaine.
\end{itemize}

\subsection*{Cadre d'organisation~:}
Encadrement dans le laboratoire de recherche en informatique (LRI)
dans l'équipe VALS\@.
Directeur Prof.\ B. Wolff (HDR), Co-encadrement : Dr.\ Safouan Taha.
Financement de Thèse: SystemX suivant la convention.
On note que l'équipe VALS va être transférée
dans le nouveau laboratoire LMF (également Université Paris-Saclay)
pendant la durée de la thèse ;
on estime que ce transfert de l'organisation n'aura aucun impact
sur le déroulement de la thèse et les engagements contractuels.

\bibliographystyle{unsrt}
\bibliography{biblio}

\end{document}
